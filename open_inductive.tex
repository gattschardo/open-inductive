\documentclass{llncs}

\usepackage[T1]{fontenc}
\usepackage[utf8]{inputenc}
\usepackage[english]{babel}
\usepackage{microtype}
\usepackage{paralist}
\usepackage{hyperref}

\usepackage[safe]{tipa} % for \textlambda
\usepackage{amsmath}
\usepackage{mathpartir}


\urlstyle{sf}
\makeatletter
% Inspired by http://anti.teamidiot.de/nei/2009/09/latex_url_slash_spacingkerning/
% but slightly less kern and shorter underscore
\let\UrlSpecialsOld\UrlSpecials
\def\UrlSpecials{\UrlSpecialsOld\do\/{\Url@@slash}\do\_{\Url@@underscore}}%
\def\Url@@slash{\@@ifnextchar/{\kern-.11em\mathchar47\kern-.2em}%
   {\kern-.0em\mathchar47\kern-.08em\penalty\UrlBigBreakPenalty}}
\def\Url@@underscore{\nfss@@text{\leavevmode \kern.06em\vbox{\hrule\@@width.3em}}}
\makeatother


% For the launchbury stuff, copied from elsewhere
% Syntax
\newcommand{\sApp}[2]{#1\;#2}
\newcommand{\sLam}[2]{\text{\textlambda} #1.\, #2}
\newcommand{\sLet}[2]{\text{\textsf{let}}\ #1\ \text{\textsf{in}}\ #2}
\newcommand{\sred}[4]{#1 : #2 \Downarrow #3 : #4}
% 'DOWNWARDS TRIPLE ARROW' (U+290B)
\newcommand{\ssred}[4]{#1 : #2 \mathrel{\rotatebox[origin=c]{90}{$\Lleftarrow$}} #3 : #4}
\newcommand{\sRule}[1]{\text{{\textsc{#1}}}}
\newcommand{\dom}[1]{\mathsf{dom}\;#1}



\title{Open Inductive Predicates}
\author{Joachim Breitner \and Richard Molitor}
\institute{Karlsruhe Institute of Technology}

\begin{document}

\maketitle

\begin{abstract}
TODO
\end{abstract}

\section{Introduction}

Inductively defined predicates are the bread and butter of many formal works, from programming languages semantics over types and logics to abstract algebra.

As a running example, consider the inductively defined semantics relation $\Downarrow$ for a lambda calculus, given by the introduction rules in Figure~\ref{fig:launchbury}, and the theorem $\sred\Gamma e \Delta v \implies \dom\Gamma \subseteq \dom\Delta$. We would prove this by induction on the deriviation of $\Downarrow$, which would require proving one case per induction rule.

\begin{figure}[b]
\begin{mathpar}
\inferrule
{ }
{\sred{\Gamma}{\sLam xe}{\Gamma}{\sLam xe}}
\sRule{Lam}
\and
\inferrule
{\sred{\Gamma}e{\Delta}{\sLam y e'}\\ \sred{\Delta}{e'[x/y]}{\Theta}{v}}
{\sred\Gamma{\sApp e x}\Theta v}
\sRule{App}
\and
\inferrule
{\sred\Gamma e \Delta v}
{\sred{\Gamma, x\mapsto e} x {\Delta, x\mapsto v}{v}}
\sRule{Var}
\and
\inferrule
{\text{$x$ fresh}\\ \sred{\Gamma,x\mapsto e'} e \Delta v}
{\sred{\Gamma}{\sLet{x = e'}e} \Delta v}
\sRule{Let}
\end{mathpar}
\caption{A typical inductively defined semantics (\cite{launchbury}, simplified)}
\label{fig:launchbury}
\end{figure}


Later, we might want to discuss a slightly different semantics, i.e. one where rule $\sRule{Var}$ was swapped for
\begin{mathpar}
\inferrule
{\sred{\Gamma, x\mapsto e} e {\Delta}{v}} 
{\sred{\Gamma, x\mapsto e} x {\Delta}{v}}
\sRule{Var'}.
\end{mathpar}
If we claim that the lemma above holds for the slighly modified semantics, as well, we have to prove it.

In a pen-and-paper proof, we would not write a complet proof by induction again, as -- quite obvious to anyone regularly working with inductive proofs -- three of the four subproofs would be identical. Instead, we would only provide the proof for the case \sRule{Var'}.

But when working with an interactive theorem prover, such convenience is not easily possible: We would have to copy-and-paste the definition of $\Downarrow$ and the proof, and then adjust the rule \sRule{Var} and corresponding proof case. We might abstract the common parts into lemmas of their own, but that again requires carefully stating the proof obligations of that case as a separate lemma. In either case, maintaining such redudancy will causes headaches when further need to be made. Furthermore, this does not scale to many variants of one predicate, e.g. when adding a garbage collection rule.

Our aim is to make such convenience as in the pen-and-paper proof available to the users of interactive theorem provers. In order to do so, we formalize the intuition behind the shortcuts and introduce \emph{open inductive predicates}. An inductive proof of a theorem involving such an open predicate will be proven for one inductive case at once, so for each concrete instance of the predicate, where the set of introduction rules is finalized, those theorems are available that have been shown for all used introduction rules.

\noindent Our contributions are
\begin{itemize}
\item We introduce the theory of open inductive theorems to formalize the folklore around inductively defined predicate and their modularization.
\item We describe an implementation of open inductive theorems as a conservative extension to the interactive theorem prover Isabelle/HOL.
\end{itemize}



\section{Conclusion}

\subsubsection*{Acknowledgments}

\bibliographystyle{splncs03}
\bibliography{bib}
\end{document}
