\section{Registering Predicates}

Defining an inductive predicate is a two-step process.
%
First, the name of the predicate needs to be registered.
%
Then one or more defining introduction rules can be associated with it.
%

\paragraph{Predicate Registration.} The command to register a new inductive predicate in the current theory is \textbf{open\_inductive}.
%
Its syntax is:
%

\begin{cmd} \label{cmdopen}
\textbf{open\_inductive} pred-name[::type]
\end{cmd}
Here, \emph{pred-name} can be any valid identifier.
%
The type annotation is optional.
%
Successfully registering a predicate will result in a diagnostic message:

\begin{msg}
\emph{Two variants of this message exist, depending on whether a type annotation was given:}
\begin{itemize}
\item Registered open inductive predicate \emph{name} without type.
\item Registered open inductive predicate \emph{name} with type \emph{type}.
\end{itemize}
\end{msg}

It is legal to register the same predicate multiple times.
%
This can be used when a predicate has some introduction rules defined in one file and is to be extended in another to remind the reader of its presence.
%
If a type was specified, it cannot be changed later however.
%
This leads to an error with the message:
%

\begin{err}
Can't re-type predicate, was \emph{old-type} would become \emph{new-type}.
\end{err}

Also, if a type was specified the predicate cannot be registered later without a type.
%
This leads to an error with the message:
%

\begin{err}
Can't delete type from predicate, was \emph{type}.
\end{err}

\paragraph{Introduction Rules.}

The command to associate an introduction rule with an open predicate is \textbf{add\_intro}, its syntax is:
%

\begin{cmd} \label{cmdintro}
\textbf{add\_intro} pred-name intro-name\textbf{:} term
\end{cmd}
Here, \emph{pred-name} refers to a previously registered predicate, \emph{intro-name} is a new name for the introduction rule and \emph{term} is an inner syntax expression containing the introduction rule.
%

Successful registration of an introduction rules results in a diagnostic message:
%

\begin{msg}
Registered introduction rule \emph{rule-name}: \emph{term} for \emph{pred-name}
\end{msg}

The predicate that \textbf{add\_intro} refers to as \emph{pred-name} needs to be registered beforehand using Command \ref{cmdopen}.
%
Using \textbf{add\_intro} with an unregistered predicate results in an error with the message:
%

\begin{err}
No such open inductive predicate: \emph{pred-name}
\end{err}

The defining term is checked for syntax and internal consistency by the parser.
%
The occurrences of the predicate in the rule are also checked to match the type of the predicate at the point of registration (if specified).
%
Additionally they are checked structurally by \textbf{add\_intro}, so it is impossible to add a rule that cannot be used as introduction rule (i.e. when the conclusion does not contain the predicate).
%

\section{Registering Theorems}

Using an open inductive predicate, theorems can be proved by induction with a separate proof for each introduction rule.
%
Like predicates, theorems need to be registered as open theorems first.
%
This is accomplished by the command \textbf{open\_theorem}, its syntax is:
%

\begin{cmd} \label{cmdthm}
\textbf{open\_theorem} theorem-name \textbf{shows} term
\end{cmd}

Here, \emph{theorem-name} is the name with which the theorem is later installed in the theory, \emph{term} is the proposition written in as an inner syntax term.
%

There is no need to explicitly specify which open inductive predicates occur in the theorem.
%
The package parses the term, finds them and reports them back to the user.
%
The message after successful registration is:
%

\begin{msg}
Declared open theorem \emph{theorem-name} as \emph{term} on \emph{predicate-name(s)}.
\end{msg}

Here, \emph{predicate-name(s)} is the list of open inductive predicates that are found in the theorem.
%
Again, all occurring predicates are checked to be compatible with their annotated type (if specified).
%

\section{Proving Theorems}

A registered open theorem can now be proved by induction.
%
The case for each introduction rule is a separate proof.
%
The command to begin a proof is \textbf{show\_open}, its syntax is:
%

\begin{cmd} \label{cmdshow}
\textbf{show\_open} \textit{theorem-name} \textbf{for} \textit{intro-name} [\textbf{assumes} \textit{intro-name(s)} \dots]
\end{cmd}

Here, \emph{theorem-name} refers to the theorem that is to be proved and \emph{intro-name} refers to the name of the introduction rule for which the inductive case is to be instantiated.
%
The optional list of \emph{intro-name(s)} are names of additional introduction rules that shall be added to the proof context as assumptions.
%
At this point, the type of the predicate is matched with its inferred type in the introduction rule.
%
If this fails a type unification error is printed.
%

The theorem needs to be registered first with Command \ref{cmdthm}.
%
Referring to an unregistered theorem leads to an error with the message:
%

\begin{err}
No such theorem: \emph{theorem-name}
\end{err}

Likewise, the introduction rule needs to be added to the predicate occurring in the proposition using Command \ref{cmdintro}.
%
Referring to an unregistered theorem leads to an error with the message:
%

\begin{err}
No introduction rule named \emph{intro-name} in \textit{pred-name(s)}
\end{err}

If no error occurs, Isabelle switches to proof mode and the theorem can now be proved for the selected introduction rule.
%
The proof context is enriched with the selected introduction rule(s), they are named \emph{pred-name.intro-name}, just as they would be when produced by the Inductive package.
%

Following a successful proof no theorem is installed in the current theory.
%
The package saves the proof internally and only exports the complete theorems, not the case for each introduction rule.
%

\paragraph{Non-inductive proofs.} It also possible to prove theorems that do not require induction. In this case the invocation is:

\begin{cmd} \label{cmdshowdirect}
\textbf{show\_open} \textit{theorem-name} [\textbf{assumes} \textit{intro-name(s)} \dots]
\end{cmd}

No induction theorem is applied to the proposition.
%
As with inductive proofs, the proof context is enriched with the selected introduction rules, registered with their usual names.
%

\section{Closing Predicates}

To get theory-level definitions of the predicates and theorems with proofs assembled from the cases, the \textbf{close\_inductive} command is used.
%
Its syntax is:
%

\begin{cmd}
\textbf{close\_inductive} pred-name \textbf{assumes} intro-name [\textbf{and} intro-name \dots] \textbf{for} close-name
\end{cmd}

The predicate named \emph{pred-name} will be created with the listed introduction rules with the theory-level name \emph{close-name}.
%
Here, \emph{close-name} can be identical to \emph{pred-name}.
%

The predicate referred to by \emph{pred-name} needs to be registered using Command \ref{cmdopen}.
%
Referring to an unregistered predicate leads to an error with the message:
%

\begin{err}
Undefined open predicate: \emph{pred-name}
\end{err}

Likewise, each of the rules listed as \emph{intro-name} needs to be registered in the correct predicate using Command \ref{cmdintro}.
%
Referring to an introduction rule not added to the predicate leads to an error with the message:
%

\begin{err}
No introduction rule name \emph{intro-name} defined in open predicate \emph{pred-name}
\end{err}

If no error occurs, a message is printed:
\begin{msg}
Closing inductive predicate \emph{pred-name} with \emph{intro-name(s)} as \emph{close-name}.
Candidates for closing: \emph{theorem-name(s)}
\end{msg}

The candidates for closing are open theorems (registered with Command \ref{cmdthm}) that contain the specified predicate.
%
If a direct proof given using Command \ref{cmdshowdirect} exists for a theorem, this proof is preferred.
%
If such a proof cannot be used because of introduction rules that are assumed but not present in the close predicate, a warning is printed:

\begin{msg} \label{msgassm}
Cannot close open theorem \emph{theorem-name}, missing introduction rules: \emph{intro-name(s)}
\end{msg}

If no direct proof exists, the inductive proofs are checked for completeness.
%
If any of these theorems cannot be closed because of missing proofs (given using Command \ref{cmdshow}), a warning is printed:
%

\begin{msg}
Cannot close open theorem \emph{theorem-name}, missing proofs for \emph{intro-name(s)}
\end{msg}

If all inductive proofs exist, but contain assumptions that cannot be satisfied, Message~\ref{msgassm} is printed here, too.
%
Open theorems for which a complete proof can be assembled are registered in the theory by their names.
%
For successful registration a message is printed:

\begin{msg}
Installing \emph{name}: \emph{term}
\end{msg}

Here, \emph{term} is the (now proved) proposition.
%
The name is of the form \emph{theorem-name\_pred-name} with \emph{theorem-name} being the name of the registered open theorem and \emph{close-name} the name of the predicate chosen with \textbf{close\_inductive}.
%
This is to avoid name clashes when closing multiple variants of the same predicate that have proofs for the same theorems.
%
